\documentclass{book}
\usepackage[times,hyper]{Rd}
\begin{document}
\HeaderA{SkewedHyperbolicDistribution}{Skewed Hyperbolic Student's t-Distribution}{SkewedHyperbolicDistribution}
\aliasA{dskewhyp}{SkewedHyperbolicDistribution}{dskewhyp}
\aliasA{pskewhyp}{SkewedHyperbolicDistribution}{pskewhyp}
\aliasA{qskewhyp}{SkewedHyperbolicDistribution}{qskewhyp}
\aliasA{rskewhyp}{SkewedHyperbolicDistribution}{rskewhyp}
\keyword{distribution}{SkewedHyperbolicDistribution}
\begin{Description}\relax
Density function, distribution function, quantiles and random number
generation for the Generalised Hyperbolic Skew Student's
t-Distribution, with parameters \eqn{\beta}{beta} (skewness),
\eqn{\delta}{delta} (peakedness), \eqn{\mu}{mu} (location) and
\eqn{\nu}{nu} (shape).
\end{Description}
\begin{Usage}
\begin{verbatim}
dskewhyp(x, mu = 0, delta = 1, beta = 1, nu = 1, param = c(mu, delta,
  beta, nu), log = FALSE, tolerance = .Machine$double.eps^0.5)
pskewhyp(q, mu = 0, delta = 1, beta = 1, nu = 1,param =
  c(mu,delta,beta,nu), log = FALSE, lower.tail = TRUE, small = 10^(-6),
  tiny = 10^(-10), subdivisions = 100, accuracy = FALSE, ...)
qskewhyp(p, mu = 0, delta = 1, beta = 1, nu = 1, param =
  c(mu,delta,beta,nu), small = 10^(-6), tiny = 10^(-10), deriv = 0.3,
  nInterpol = 100, subdivisions = 100, ...)
rskewhyp(n, mu = 0, delta = 1, beta = 1, nu = 1, param =
  c(mu,delta,beta,nu), log = FALSE)
\end{verbatim}
\end{Usage}
\begin{Arguments}
\begin{ldescription}
\item[\code{x,q}] Vector of quantiles.
\item[\code{p}] Vector of probabilities.
\item[\code{n}] Number of random variates to be generated.
\item[\code{mu}] Location parameter \eqn{\mu}{mu}, default is 0.
\item[\code{delta}] Peakedness parameter \eqn{\delta}{delta}, default is 1.
\item[\code{beta}] Skewness parameter \eqn{\beta}{beta}, default is 1.
\item[\code{nu}] Shape parameter \eqn{\nu}{nu}, default is 1.
\item[\code{param}] Specifying the parameters as a vector of the form
\code{c(mu, delta, beta, nu)}.
\item[\code{log}] Logical; if \code{log = TRUE}, probabilities are given as
log(p).
\item[\code{lower.tail}] Logical; if \code{lower.tail = TRUE}, the cumulative
density is taken from the lower tail.
\item[\code{tolerance}] Specified level of tolerance when checking if
parameter beta is equal to 0.
\item[\code{small}] Size of a small difference between the distribution
function and 0 or 1.
\item[\code{tiny}] Size of a tiny difference between the distribution
function and 0 or 1.
\item[\code{subdivisions}] The maximum number of subdivisions used to
integrate the density and determine the accuracy of the distribution
function calculation.
\item[\code{accuracy}] Logical; if \code{accuracy = TRUE}, accuracy
calculated by \code{\LinkA{integrate}{integrate}} to try and determine the
accuracy of the distribution function calculation.
\item[\code{deriv}] Value between 0 and 1 which determines the point at which
the value of the derivative becomes substantial compared to its
maximal value, see \bold{Details}.
\item[\code{nInterpol}] Number of points used in \code{qskewhyp} for cubic
spline interpolation of the distribution function.
\item[\code{...}] Passes additional arguments to \code{\LinkA{integrate}{integrate}}.
\end{ldescription}
\end{Arguments}
\begin{Details}\relax
Users may either specify the values of the parameters individually or
as a vector. If both forms are specified, then the values specified by
the vector \code{param} will overwrite the other ones.

The density function is

\deqn{f(x)=\frac{2^{(\frac{1-\nu}{2})}\delta^\nu
|\beta|^{(\frac{\nu+1}{2})}
K_{(\frac{\nu+1}{2})}\sqrt{(\beta^2(\delta^2+(x-\mu)^2))}
exp(\beta(x-\mu))}{\Gamma(\frac{\nu}{2})\sqrt{(\pi)}
\sqrt{(\delta^2+(x-\mu)^2)^{(\frac{\nu+1}{2})}}}}{f(x) = (2^((1-nu)/2) delta^nu abs(beta)^((nu+1)/2)
K_((nu+1)/2)(sqrt(beta^2 (delta^2+(x-mu)^2)) )
exp(beta (x-mu)))/ (gamma(nu/2) sqrt(pi)
(sqrt(delta^2+(x-mu)^2))^((nu+1)/2))}

when \eqn{\beta \ne 0}{beta != 0}, and

\deqn{ f(x)=\frac{\Gamma(\frac{\nu+1}{2})}{\sqrt{(\pi)}\delta
\Gamma(\frac{\nu}{2})}\left(1+\frac{(x-\mu)^2}{\delta^2}\right)^
\frac{-(\nu+1)}{2} }{f(x)=(gamma((nu+1)/2)/(sqrt(pi) 
delta gamma(nu/2)))(1+((x-mu)^2)/(delta^2))^(-(nu+1)/2)}

when \eqn{\beta = 0}{beta = 0},
where \eqn{K_{nu}(.)}{K_nu(.)} is the modified Bessel function of the third
kind with order nu, and \eqn{\Gamma}{gamma} is the gamma function.

\code{pskewhyp} breaks the real line in to 8 regions in order to
determine the integral of \code{dhyperb}. The breakpoints determining
the regions are calculated by \code{\LinkA{skewhypBreaks}{skewhypBreaks}}, based on the
values of \code{small}, \code{tiny}, and \code{deriv}.

The inner area is divided into two regions above and two below the
mode. The breakpoint that divides these is calculated to be where the
derivative of the density function is \code{deriv} times the value of
the maximum derivative on that side of the mode. In the extreme tails
of the distribution where the probability is \code{tiny} the
probability is taken to be zero. In the remaining regions the integral
of the density is calculated using the numerical routine
\code{\LinkA{safeIntegrate}{safeIntegrate}} (a wrapper for
\code{\LinkA{integrate}{integrate}}).

\code{qhyperb} Used the same breakup of the real line as
\code{pskewhyp}. For quantiles that fall in the two extreme regions
the quantile is returned as \code{Inf} or \code{-Inf} as
appropriate. In the remaining regions \code{splinefun} is used to fit
values of the distribution function calulated by \code{pskewhyp}. The
quantiles are then found by the \code{uniroot} function.
\end{Details}
\begin{Value}
\code{dskewhyp} gives the density function, \code{pskewhyp} gives the
distribution function, \code{qskewhyp} gives the quantile function and
\code{rskewhyp} generates random variates.

An estimate of the accuracy of the approximation to the distribution
function can be found by setting \code{accuracy=TRUE} in the call to
\code{pskewyhp} which returns a list with components \code{value} and
\code{error}.
\end{Value}
\begin{Author}\relax
David Scott \email{d.scott@auckland.ac.nz}, Fiona Grimson
\end{Author}
\begin{References}\relax
Aas, K. and Haff, I. H. (2006).
The generalised hyperbolic skew Student's \emph{t}-distribution,
\emph{Journal of Financial Econometrics}, \bold{4}, 275--309.
\end{References}
\begin{SeeAlso}\relax
\code{\LinkA{safeIntegrate}{safeIntegrate}}, \code{\LinkA{integrate}{integrate}} for
its shortfalls, \code{\LinkA{skewhypBreaks}{skewhypBreaks}}
\end{SeeAlso}

\HeaderA{qqskewhyp}{Skewed Hyperbolic Student's t-Distribution Quantile-Quantile and
Percent-Percent Plots}{qqskewhyp}
\aliasA{ppskewhyp}{qqskewhyp}{ppskewhyp}
\keyword{distribution}{qqskewhyp}
\begin{Description}\relax
\code{qqskewhyp} produces a skewed hyperbolic t-distribution Q-Q plot
of the values in \code{y}, \code{ppskewhyp} produces a skewed
hyperbolic t-distribution P-P (percent-percent) plot or probability
plot of the values in \code{y}.
Graphical parameters may be given as arguments to \code{qqskewhyp} and
\code{ppskewhyp}.
\end{Description}
\begin{Usage}
\begin{verbatim}
qqskewhyp(y, mu = 0, delta = 1, beta = 1, nu = 1, param = c(mu, delta,
beta, nu), main = "Skewed Hyperbolic Student's-t QQ Plot", xlab =
"Theoretical Quantiles", ylab = "Sample Quantiles", plot.it = TRUE, line
= TRUE, ...)
ppskewhyp(y, beta = NULL, delta = NULL, mu = NULL, nu = NULL, param =
c(mu, delta, beta, nu), main = "Skewed Hyperbolic Student's-t P-P Plot",
xlab = "Uniform Quantiles", ylab =
"Probability-integral-transformed Data", plot.it = TRUE, line = TRUE, ...)

\end{verbatim}
\end{Usage}
\begin{Arguments}
\begin{ldescription}
\item[\code{y}] The sample data.
\item[\code{mu}] Location parameter \eqn{\mu}{mu}, default is 0.
\item[\code{delta}] Peakedness parameter \eqn{\delta}{delta}, default is 1.
\item[\code{beta}] Skewness parameter \eqn{\beta}{beta}, default is 1.
\item[\code{nu}] Shape parameter \eqn{\nu}{nu}, default is 1.
\item[\code{param}] Specifying the parameters as a vector of the form
\code{c(mu, delta, beta, nu)}.
\item[\code{main,xlab,ylab}] Plot labels.
\item[\code{plot.it}] Logical; if \code{plot.it = TRUE} the results will be plotted.
\item[\code{line}] Logical; if \code{line = TRUE} a line is added through the
origin with unit slope.
\item[\code{...}] Further graphical parameters.
\end{ldescription}
\end{Arguments}
\begin{Details}\relax
Users may either specify the values of the parameters individually or
as a vector. If both forms are specified, then the values specified by
the vector \code{param} will overwrite the other ones.
\end{Details}
\begin{Value}
For \code{qqskewhyp} and \code{ppskewhyp}, a list with components:
\begin{ldescription}
\item[\code{x}] The x coordinates of the points to be plotted.
\item[\code{y}] The y coordinates of the points to be plotted.
\end{ldescription}
\end{Value}
\begin{Author}\relax
David Scott \email{d.scott@auckland.ac.nz}, Fiona Grimson
\end{Author}
\begin{SeeAlso}\relax
\code{\LinkA{ppoints}{ppoints}}, \code{\LinkA{qqplot}{qqplot}}, \code{\LinkA{dskewhyp}{dskewhyp}}
\end{SeeAlso}

\HeaderA{SkewedHyperbolic-package}{The Package 'SkewedHyperbolic': Summary Information}{SkewedHyperbolic.Rdash.package}
\aliasA{SkewedHyperbolic}{SkewedHyperbolic-package}{SkewedHyperbolic}
\aliasA{skewhyp}{SkewedHyperbolic-package}{skewhyp}
\keyword{package}{SkewedHyperbolic-package}
\begin{Description}\relax
This package provides a collection of functions for working with the
skewed hyperbolic Student's t-distribution.

Functions are provided for the density function (\code{dskewhyp}),
distribution function (\code{pskewhyp}), quantiles (\code{qskewhyp}) and
random number generation (\code{rskewhyp}). There are also functions
that fit the distribution to data (\code{skewhypFit}). The mean,
variance, skewness, kurtosis and mode can be found usig the functions
\code{skewhypMean}, \code{skewhypVar}, \code{skewhypSkew},
\code{skewhypKurt} and \code{skewhypMode} respectively. To assess
goodness of fit, there are also functions to generate a Q-Q plot
(\code{qqskewhyp}) and a P-P plot (\code{ppskewhyp}). S3 methods
\code{print}, \code{plot} and \code{summary} are provided for the output
of \code{skwewhypFit}.

\Tabular{ll}{
Package: & SkewedHyperbolic\\
Type: & Package\\
Version: & 0.1-1\\
Date: & 2009-08-28\\
License: & GPL(>=2)\\
LazyLoad: & yes\\
}
\end{Description}
\begin{Author}\relax
David Scott \email{d.scott@auckland.ac.nz}, Fiona Grimson
\end{Author}
\begin{References}\relax
Aas, K. and Haff, I. H. (2006).
The generalised hyperbolic skew Student's \emph{t}-distribution,
\emph{Journal of Financial Econometrics}, \bold{4}, 275--309.
\end{References}
\begin{SeeAlso}\relax
\code{\LinkA{dskewhyp}{dskewhyp}}, \code{\LinkA{skewhypMean}{skewhypMean}},
\code{\LinkA{skewhypFit}{skewhypFit}}, \code{\LinkA{skewhypFitStart}{skewhypFitStart}},
\code{\LinkA{skewhypBreaks}{skewhypBreaks}}, \code{\LinkA{qqskewhyp}{qqskewhyp}},
\code{\LinkA{HyperbolicDist}{HyperbolicDist}}.
\end{SeeAlso}

\HeaderA{skewhypBreaks}{Break points for the Skewed Hyperbolic Student's t-Distribuiton}{skewhypBreaks}
\aliasA{ddskewhyp}{skewhypBreaks}{ddskewhyp}
\aliasA{skewhypCalcRange}{skewhypBreaks}{skewhypCalcRange}
\keyword{distribution}{skewhypBreaks}
\begin{Description}\relax
Utility routines that calculate suitable breakpoints for use in
determining the distribution function, and the derivative of the
density function.
\end{Description}
\begin{Usage}
\begin{verbatim}
skewhypBreaks(mu = 0, delta = 1, beta = 1, nu = 1, param =
  c(mu, delta, beta, nu), small = 10^(-6), tiny = 10^(-10),
  deriv = 0.3, ...)
skewhypCalcRange(mu = 0, delta = 1, beta = 1, nu = 1,
  param = c(mu,delta,beta,nu), tol= 10^(-5), ...)
ddskewhyp(x, mu = 0, delta = 1, beta = 1, nu = 1, param =
  c(mu,delta,beta,nu),log = FALSE, tolerance =
  .Machine$double.eps ^ 0.5)
\end{verbatim}
\end{Usage}
\begin{Arguments}
\begin{ldescription}
\item[\code{x}] Vector of quantiles
\item[\code{mu}] Location parameter \eqn{\mu}{mu}, default is 0.
\item[\code{delta}] Peakedness parameter \eqn{\delta}{delta}, default is 1.
\item[\code{beta}] Skewness parameter \eqn{\beta}{beta}, default is 1.
\item[\code{nu}] Shape parameter \eqn{\nu}{nu}, default is 1.
\item[\code{param}] Specifying the parameters as a vector of the form
\code{c(mu, delta, beta, nu)}.
\item[\code{small}] Size of a small difference between the distribution
function and 0 or 1.
\item[\code{tiny}] Size of a tiny difference between the distribution
function and 0 or 1.
\item[\code{deriv}] Value between 0 and 1 which determines the point at which
the value of the derivative becomes substantial compared to its
maximal value, see \bold{Details}.
\item[\code{log}] Logical; if \code{log = TRUE}, probabilities are given as log(p).
\item[\code{tolerance}] Specified level of tolerance when checking if parameter beta
is equal to 0 in \code{ddskewhyp}.
\item[\code{tol}] Tolerance used in \code{skewhypCalcRange}, see
\bold{Details}.
\item[\code{...}] Passes additional arguments to \code{\LinkA{uniroot}{uniroot}}.
\end{ldescription}
\end{Arguments}
\begin{Value}
\code{ddskewhyp} gives the derivative of \code{\LinkA{dskewhyp}{dskewhyp}}.

\code{skewhypBreaks} returns a list with components:
\begin{ldescription}
\item[\code{xTiny}] Value such that the probabilities to the left are less
than \code{tiny}.
\item[\code{xSmall}] Value such that probabilities to the left are less
than \code{small}.
\item[\code{lowBreak}] Point to the left of the mode such that the
derivative of the density at that point is \code{deriv} times its
maximum value on that side of the mode.
\item[\code{highBreak}] Point to the right of the mode such that the derivative
of the density at that point is \code{deriv} times the value of the
maximum value on that side of the mode.
\item[\code{xLarge}] Value such that the probabities to the right are less than
\code{xSmall}.
\item[\code{xHuge}] Value such that probability to the right is less than
\code{tiny}.
\item[\code{modeDist}] The mode of the given skewed hyperbolic distribution,
calculated by \code{\LinkA{skewhypMode}{skewhypMode}}.
\end{ldescription}


\code{skewhypCalcRange} returns the quantile values at the
endpoints of the region of the density function where the probablity
is greater than \code{tol}.
\end{Value}
\begin{Author}\relax
David Scott \email{d.scott@auckland.ac.nz}, Fiona Grimson
\end{Author}
\begin{SeeAlso}\relax
\code{\LinkA{uniroot}{uniroot}}, \code{\LinkA{dskewhyp}{dskewhyp}}, \code{\LinkA{skewhypMode}{skewhypMode}}
\end{SeeAlso}

\HeaderA{skewhypFit}{Fit the Skewed Hyperbolic Student's t-Distribution to Data}{skewhypFit}
\aliasA{plot.skewhypFit}{skewhypFit}{plot.skewhypFit}
\aliasA{print.skewhypFit}{skewhypFit}{print.skewhypFit}
\aliasA{summary.skewhypFit}{skewhypFit}{summary.skewhypFit}
\keyword{distribution}{skewhypFit}
\begin{Description}\relax
Fits a skewed hyperbolic t-distribution to given data.
Displays the histogram, log-histogram (both with fitted densities),
Q-Q plot and P-P plot for the fit which has maximum likelihood.
\end{Description}
\begin{Usage}
\begin{verbatim}
skewhypFit(x, freq = NULL, breaks = NULL, startValues = "LA",
  paramStart = NULL, method = "Nelder-Mead", hessian = TRUE,
  plots = TRUE, printOut = TRUE, controlBFGS = list(maxit = 200),
  controlNM = list(maxit = 1000), maxitNLM = 1500, ...)
## S3 method for class 'skewhypFit':
plot(x, which = 1:4, plotTitles = paste(c(
"Histogram of ", "Log-Histogram of ", "Q-Q Plot of ", "P-P Plot of "),
x$obsName, sep = ""), ask = prod(par("mfcol")) < length(which) &&
dev.interactive(), ...)
## S3 method for class 'skewhypFit':
print(x,digits = max(3, getOption("digits") - 3),...)
## S3 method for class 'skewhypFit':
summary(object,...)
\end{verbatim}
\end{Usage}
\begin{Arguments}
\begin{ldescription}
\item[\code{x}] Data Vector for \code{skewhypFit}. Object of class
\code{"skewhypFit"} for \code{plot.skewhypFit},
\code{print.skewhypFit} and \code{summary.skewhypFit}.
\item[\code{freq}] Vector of weights with length equal to length of x.
\item[\code{breaks}] Breaks for histogram, defaults to those generated by
\code{hist(x, plot=FALSE, right=FALSE).}If \code{startValues = "LA"}
then 30 breaks are used by default.
\item[\code{startValues}] Code giving the method of determining starting
values for finding the maximum likelihood estimates of the
parameters.
\item[\code{paramStart}] If \code{startValues = "US"} the user must specify a
vector of starting parameter values in the form \code{c(mu, delta,
      beta, nu)}.
\item[\code{method}] Different optimisation methods to consider, see
\bold{Details}.
\item[\code{hessian}] Logical; if \code{hessian = TRUE} the value of the
hessian is returned.
\item[\code{plots}] Logical; if \code{plots = TRUE} the histogram,
log-histogram, Q-Q and P-P plots are printed.
\item[\code{printOut}] Logical; if \code{printOut = TRUE} results of the
fitting are printed.
\item[\code{controlBFGS}] A list of control parameters for \code{optim} when
using the \code{"BFGS"} optimisation.
\item[\code{controlNM}] A list of control parameters for \code{optim}
when using the \code{"Nelder-Mead"} optimisation.
\item[\code{maxitNLM}] A positive integer specifying the maximum number of
iterations when using the \code{"nlm"} optimisation.
\item[\code{which}] If a subset of plots is required, specify a subset of the
numbers \code{1:4}.
\item[\code{plotTitles}] Titles to appear above the plots.
\item[\code{ask}] Logical; if \code{TRUE} the user is asked before plot
change, see \code{\LinkA{par}{par}(ask = .)}.
\item[\code{digits}] Desired number of digits when the object is printed.
\item[\code{object}] Object must be of class "skewhypFit"
\item[\code{...}] Passes arguments to \code{\LinkA{optim}{optim}}, \code{\LinkA{nlm}{nlm}},
\code{\LinkA{hist}{hist}},\code{\LinkA{logHist}{logHist}},
\code{\LinkA{qqskewhyp}{qqskewhyp}} and \code{\LinkA{ppskewhyp}{ppskewhyp}}.
\end{ldescription}
\end{Arguments}
\begin{Details}\relax
\code{startValues} can be either \code{"US"}(User-supplied) or
\code{"LA"} (Linear approximation)
If \code{startValues = "US"} then a value for \code{paramStart} must be
supplied. For the details concerning the use of \code{startValues}
and \code{paramStart} see \code{\LinkA{skewhypFitStart}{skewhypFitStart}}.

The three optimisation methods currently available are:
\Itemize{
\item[\code{"BFGS"}] Uses the quasi-Newton method \code{"BFGS"} as
documented in \code{\LinkA{optim}{optim}}.
\item[\code{"Nelder-Mead"}] Uses an implementation of the Nelder and
Mead method as documented in \code{\LinkA{optim}{optim}}.
\item[\code{"nlm"}] Uses the \code{\LinkA{nlm}{nlm}} function in R.
}
For the details of how to pass control information using
\code{\LinkA{optim}{optim}} and \code{\LinkA{nlm}{nlm}}, see \code{\LinkA{optim}{optim}} and
\code{\LinkA{nlm}{nlm}.}
\end{Details}
\begin{Value}
\code{skewhypFit} returns a list with components:
\begin{ldescription}
\item[\code{param}] A vector giving the maximum likelihood estimates of the
parameters in the form \code{c(mu, delta, beta, nu)}.
\item[\code{maxLik}] The value of the maximised log-likelihood.
\item[\code{hessian}] If \code{hessian} was set to \code{TRUE}, the value of
the hessian, not present otherwise.
\item[\code{method}] Optimisation method used.
\item[\code{conv}] Convergence code. See \code{\LinkA{optim}{optim}} or
\code{\LinkA{nlm}{nlm}} for details.
\item[\code{iter}] Number of iterations of optimisation routine.
\item[\code{x}] The data used to fit the distribution.
\item[\code{xName}] Character stirng with the actual \code{x} argument name.
\item[\code{paramStart}] Starting values of the parameters returned by
\code{\LinkA{skewhypFitStart}{skewhypFitStart}}.
\item[\code{svName}] Name of the method used to find starting values.
\item[\code{startValues}] Acronym of method used to find starting values.
\item[\code{breaks}] Cell boundaries found by a call to \code{\LinkA{hist}{hist}}.
\item[\code{midpoints}] The cell midpoints found by a call to
\code{\LinkA{hist}{hist}}.
\item[\code{empDens}] The estimated density found by a call to
\code{\LinkA{hist}{hist}} if \code{startValues = "US"} or
\code{\LinkA{density}{density}} if \code{startValues = "LA"}.
\end{ldescription}
\end{Value}
\begin{Author}\relax
David Scott \email{d.scott@auckland.ac.nz}, Fiona Grimson
\end{Author}
\begin{SeeAlso}\relax
\code{\LinkA{optim}{optim}}, \code{\LinkA{nlm}{nlm}}, \code{\LinkA{par}{par}},
\code{\LinkA{hist}{hist}}, \code{\LinkA{logHist}{logHist}},
\code{\LinkA{qqskewhyp}{qqskewhyp}},
\code{\LinkA{ppskewhyp}{ppskewhyp}} and \code{\LinkA{skewhypFitStart}{skewhypFitStart}}.
\end{SeeAlso}

\HeaderA{skewhypFitStart}{Find Starting Values for Fittting a Skewed Hyperbolic Student's
t-Distribution}{skewhypFitStart}
\aliasA{skewhypFitStartLA}{skewhypFitStart}{skewhypFitStartLA}
\keyword{distribution}{skewhypFitStart}
\begin{Description}\relax
Finds starting values for input to a maximum likelihood routine for
fitting a skewed hyperbolic t-distribution to data.
\end{Description}
\begin{Usage}
\begin{verbatim}
skewhypFitStart(x, breaks = NULL, startValues = "LA", paramStart = NULL)
skewhypFitStartLA(x, breaks = NULL)
\end{verbatim}
\end{Usage}
\begin{Arguments}
\begin{ldescription}
\item[\code{x}] Data vector.
\item[\code{breaks}] Breaks for histogram. If missing defaults to those
generated by \code{hist(x, right = FALSE, plot =FALSE)}. If
\code{startValues = "LA"} then 30 breaks are used by default.
\item[\code{startValues}] Code giving the method of determining starting
values for finding the maximum likelihood estimates of the
parameters.
\item[\code{paramStart}] If \code{startValues = "US"} the user must specify a
vector of starting parameter values in the form \code{c(mu, delta,
      beta, nu)}.
\end{ldescription}
\end{Arguments}
\begin{Details}\relax
\code{startValues} can be either \code{"US"}(User-supplied) or
\code{"LA"} (Linear approximation).

If \code{startValues = "US"} then a value for \code{paramStart} must be
supplied.

If \code{startValues = "LA"} a linear approximation is made to the
log-density in each of the tails, from which the estimates for
\eqn{\nu}{nu} and \eqn{\beta}{beta} are found. The remaining two
parameters, \eqn{\delta}{delta} and \eqn{\mu}{mu} are found by solving
the moment equations for mean and variance.
Since the variance does not exist for values of \eqn{\nu \le 4}{nu <=
4}, the estimate of \eqn{\nu}{nu} will be at least 4.1.
\end{Details}
\begin{Value}
\code{skewhypFitStart} returns a list with components:
\begin{ldescription}
\item[\code{paramStart}] A vector of the form \code{c(mu, delta, beta, nu)}
giving the generated starting values of the parameters.
\item[\code{breaks}] The cell boundaries found by a call to \code{\LinkA{hist}{hist}}.
\item[\code{midpoints}] The cell midpoints found by a call to \code{\LinkA{hist}{hist}}.
\item[\code{empDens}] The estimated density at the midpoints found by a call
to \code{\LinkA{hist}{hist}} if \code{startValues = "US"} or
\code{\LinkA{density}{density}} if \code{startValues  = "LA"}.
\item[\code{svName}] Name of the method used to find the starting values.
\end{ldescription}
\end{Value}
\begin{Author}\relax
David Scott \email{d.scott@auckland.ac.nz}, Fiona Grimson
\end{Author}
\begin{References}\relax
Aas, K. and Haff, I. H. (2006).
The Generalised Hyperbolic Skew Student's t-Distribution.
In \emph{Jorunal of Financial Econometrics}, 26-Jan-2006
\end{References}
\begin{SeeAlso}\relax
\code{\LinkA{hist}{hist}}, \code{\LinkA{density}{density}}, \code{\LinkA{dskewhyp}{dskewhyp}},
\code{\LinkA{skewhypFit}{skewhypFit}}
\end{SeeAlso}

\HeaderA{skewhypMean}{Moments and Mode of the Skewed Hyperbolic Student's t-Distribution.}{skewhypMean}
\aliasA{skewhypKurt}{skewhypMean}{skewhypKurt}
\aliasA{skewhypMode}{skewhypMean}{skewhypMode}
\aliasA{skewhypSkew}{skewhypMean}{skewhypSkew}
\aliasA{skewhypVar}{skewhypMean}{skewhypVar}
\keyword{distribution}{skewhypMean}
\begin{Description}\relax
Functions to calculate the mean, variance, skewness and kurtosis of a
specified skewed hyperbolic t-distribution.
\end{Description}
\begin{Usage}
\begin{verbatim}
skewhypMean(mu = 0, delta = 1, beta = 1, nu = 1, param = c(mu, delta,
  beta, nu))
skewhypVar(mu = 0, delta = 1, beta = 1, nu = 1, param = c(mu, delta,
  beta, nu))
skewhypSkew(mu = 0, delta = 1, beta = 1, nu = 1, param = c(mu, delta,
  beta, nu))
skewhypKurt(mu = 0, delta = 1, beta = 1, nu = 1, param = c(mu, delta,
  beta, nu))
skewhypMode(mu = 0, delta = 1, beta = 1, nu = 1, param = c(mu, delta,
  beta, nu), tolerance = .Machine$double.eps ^ 0.5)
\end{verbatim}
\end{Usage}
\begin{Arguments}
\begin{ldescription}
\item[\code{mu}] Location parameter \eqn{\mu}{mu}, default is 0.
\item[\code{delta}] Peakedness parameter \eqn{\sigma}{sigma}, default is 1.
\item[\code{beta}] Skewness parameter \eqn{\beta}{beta}, default is 1. Negative
values give a negative skew, positive values give a positive skew.
\item[\code{nu}] Shape parameter \eqn{\nu}{\nu}, default is 1.
\item[\code{param}] Specifying the parameters as a vector of the form
\code{c(mu, delta, beta, nu)}.
\item[\code{tolerance}] A difference smaller than this value is taken to be zero.
\end{ldescription}
\end{Arguments}
\begin{Details}\relax
Users may either specify the values of the parameters individually or
as a vector. If both forms are specified, then the values specified by
the vector \code{param} will overwrite the other ones.

The moments are calculated as per formulae in Aas\&Haff(2006) and the
mode is calculated by numerical optimisation of the density function
using \code{\LinkA{optim}{optim}}.

Note that the mean does not exist when \eqn{\nu = 2}{nu = 2}, the
variance does not exist for \eqn{\nu \le 4}{nu <= 4}, the skewness does
not exist for \eqn{\nu \le 6}{nu <= 6}, and the kurtosis does not exist
for \eqn{\nu \le 8}{nu <= 8}.
\end{Details}
\begin{Value}
\code{skewhypMean} gives the mean of the skewed hyperbolic
t-distribution, \code{skewhypVar} the variance, \code{skewhypSkew} the
skewness, \code{skewhypKurt} the kurtosis and \code{skewhypMode} the
mode.
\end{Value}
\begin{Author}\relax
David Scott \email{d.scott@auckland.ac.nz}, Fiona Grimson
\end{Author}
\begin{References}\relax
Aas, K. and Haff, I. H. (2006).
The Generalised Hyperbolic Skew Student's t-Distribution.
In \emph{Jorunal of Financial Econometrics}, 26-Jan-2006
\end{References}
\begin{SeeAlso}\relax
\code{\LinkA{dskewhyp}{dskewhyp}}, \code{\LinkA{optim}{optim}}
\end{SeeAlso}

\end{document}
