\HeaderA{qqskewhyp}{Skewed Hyperbolic Student's t-Distribution Quantile-Quantile and
Percent-Percent Plots}{qqskewhyp}
\aliasA{ppskewhyp}{qqskewhyp}{ppskewhyp}
\keyword{distribution}{qqskewhyp}
\begin{Description}\relax
\code{qqskewhyp} produces a skewed hyperbolic t-distribution Q-Q plot
of the values in \code{y}, \code{ppskewhyp} produces a skewed
hyperbolic t-distribution P-P (percent-percent) plot or probability
plot of the values in \code{y}.
Graphical parameters may be given as arguments to \code{qqskewhyp} and
\code{ppskewhyp}.
\end{Description}
\begin{Usage}
\begin{verbatim}
qqskewhyp(y, mu = 0, delta = 1, beta = 1, nu = 1, param = c(mu, delta,
beta, nu), main = "Skewed Hyperbolic Student's-t QQ Plot", xlab =
"Theoretical Quantiles", ylab = "Sample Quantiles", plot.it = TRUE, line
= TRUE, ...)
ppskewhyp(y, beta = NULL, delta = NULL, mu = NULL, nu = NULL, param =
c(mu, delta, beta, nu), main = "Skewed Hyperbolic Student's-t P-P Plot",
xlab = "Uniform Quantiles", ylab =
"Probability-integral-transformed Data", plot.it = TRUE, line = TRUE, ...)

\end{verbatim}
\end{Usage}
\begin{Arguments}
\begin{ldescription}
\item[\code{y}] The sample data.
\item[\code{mu}] Location parameter \eqn{\mu}{mu}, default is 0.
\item[\code{delta}] Peakedness parameter \eqn{\delta}{delta}, default is 1.
\item[\code{beta}] Skewness parameter \eqn{\beta}{beta}, default is 1.
\item[\code{nu}] Shape parameter \eqn{\nu}{nu}, default is 1.
\item[\code{param}] Specifying the parameters as a vector of the form
\code{c(mu, delta, beta, nu)}.
\item[\code{main,xlab,ylab}] Plot labels.
\item[\code{plot.it}] Logical; if \code{plot.it = TRUE} the results will be plotted.
\item[\code{line}] Logical; if \code{line = TRUE} a line is added through the
origin with unit slope.
\item[\code{...}] Further graphical parameters.
\end{ldescription}
\end{Arguments}
\begin{Details}\relax
Users may either specify the values of the parameters individually or
as a vector. If both forms are specified, then the values specified by
the vector \code{param} will overwrite the other ones.
\end{Details}
\begin{Value}
For \code{qqskewhyp} and \code{ppskewhyp}, a list with components:
\begin{ldescription}
\item[\code{x}] The x coordinates of the points to be plotted.
\item[\code{y}] The y coordinates of the points to be plotted.
\end{ldescription}
\end{Value}
\begin{Author}\relax
David Scott \email{d.scott@auckland.ac.nz}, Fiona Grimson
\end{Author}
\begin{SeeAlso}\relax
\code{\LinkA{ppoints}{ppoints}}, \code{\LinkA{qqplot}{qqplot}}, \code{\LinkA{dskewhyp}{dskewhyp}}
\end{SeeAlso}

