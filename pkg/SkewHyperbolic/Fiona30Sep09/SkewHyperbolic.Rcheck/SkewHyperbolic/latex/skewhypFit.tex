\HeaderA{skewhypFit}{Fit the Skew Hyperbolic Student's t-Distribution to Data}{skewhypFit}
\aliasA{plot.skewhypFit}{skewhypFit}{plot.skewhypFit}
\aliasA{print.skewhypFit}{skewhypFit}{print.skewhypFit}
\aliasA{summary.skewhypFit}{skewhypFit}{summary.skewhypFit}
\keyword{distribution}{skewhypFit}
\begin{Description}\relax
Fits a skew hyperbolic t-distribution to given data.
Displays the histogram, log-histogram (both with fitted densities),
Q-Q plot and P-P plot for the fit which has maximum likelihood.
\end{Description}
\begin{Usage}
\begin{verbatim}
skewhypFit(x, freq = NULL, breaks = NULL, startValues = "LA",
  paramStart = NULL, method = "Nelder-Mead", hessian = TRUE,
  plots = TRUE, printOut = TRUE, controlBFGS = list(maxit = 200),
  controlNM = list(maxit = 1000), maxitNLM = 1500, ...)
## S3 method for class 'skewhypFit':
plot(x, which = 1:4, plotTitles = paste(c(
"Histogram of ", "Log-Histogram of ", "Q-Q Plot of ", "P-P Plot of "),
x$obsName, sep = ""), ask = prod(par("mfcol")) < length(which) &&
dev.interactive(), ...)
## S3 method for class 'skewhypFit':
print(x,digits = max(3, getOption("digits") - 3),...)
## S3 method for class 'skewhypFit':
summary(object,...)
\end{verbatim}
\end{Usage}
\begin{Arguments}
\begin{ldescription}
\item[\code{x}] Data Vector for \code{skewhypFit}. Object of class
\code{"skewhypFit"} for \code{plot.skewhypFit},
\code{print.skewhypFit} and \code{summary.skewhypFit}.
\item[\code{freq}] Vector of weights with length equal to length of x.
\item[\code{breaks}] Breaks for histogram, defaults to those generated by
\code{hist(x, plot=FALSE, right=FALSE).}If \code{startValues = "LA"}
then 30 breaks are used by default.
\item[\code{startValues}] Code giving the method of determining starting
values for finding the maximum likelihood estimates of the
parameters.
\item[\code{paramStart}] If \code{startValues = "US"} the user must specify a
vector of starting parameter values in the form \code{c(mu, delta,
      beta, nu)}.
\item[\code{method}] Different optimisation methods to consider, see
\bold{Details}.
\item[\code{hessian}] Logical; if \code{hessian = TRUE} the value of the
hessian is returned.
\item[\code{plots}] Logical; if \code{plots = TRUE} the histogram,
log-histogram, Q-Q and P-P plots are printed.
\item[\code{printOut}] Logical; if \code{printOut = TRUE} results of the
fitting are printed.
\item[\code{controlBFGS}] A list of control parameters for \code{optim} when
using the \code{"BFGS"} optimisation.
\item[\code{controlNM}] A list of control parameters for \code{optim}
when using the \code{"Nelder-Mead"} optimisation.
\item[\code{maxitNLM}] A positive integer specifying the maximum number of
iterations when using the \code{"nlm"} optimisation.
\item[\code{which}] If a subset of plots is required, specify a subset of the
numbers \code{1:4}.
\item[\code{plotTitles}] Titles to appear above the plots.
\item[\code{ask}] Logical; if \code{TRUE} the user is asked before plot
change, see \code{\LinkA{par}{par}(ask = .)}.
\item[\code{digits}] Desired number of digits when the object is printed.
\item[\code{object}] Object must be of class "skewhypFit"
\item[\code{...}] Passes arguments to \code{\LinkA{optim}{optim}}, \code{\LinkA{nlm}{nlm}},
\code{\LinkA{hist}{hist}},\code{\LinkA{logHist}{logHist}},
\code{\LinkA{qqskewhyp}{qqskewhyp}} and \code{\LinkA{ppskewhyp}{ppskewhyp}}.
\end{ldescription}
\end{Arguments}
\begin{Details}\relax
\code{startValues} can be either \code{"US"}(User-supplied) or
\code{"LA"} (Linear approximation)
If \code{startValues = "US"} then a value for \code{paramStart} must be
supplied. For the details concerning the use of \code{startValues}
and \code{paramStart} see \code{\LinkA{skewhypFitStart}{skewhypFitStart}}.

The three optimisation methods currently available are:
\Itemize{
\item[\code{"BFGS"}] Uses the quasi-Newton method \code{"BFGS"} as
documented in \code{\LinkA{optim}{optim}}.
\item[\code{"Nelder-Mead"}] Uses an implementation of the Nelder and
Mead method as documented in \code{\LinkA{optim}{optim}}.
\item[\code{"nlm"}] Uses the \code{\LinkA{nlm}{nlm}} function in R.
}
For the details of how to pass control information using
\code{\LinkA{optim}{optim}} and \code{\LinkA{nlm}{nlm}}, see \code{\LinkA{optim}{optim}} and
\code{\LinkA{nlm}{nlm}.}
\end{Details}
\begin{Value}
\code{skewhypFit} returns a list with components:
\begin{ldescription}
\item[\code{param}] A vector giving the maximum likelihood estimates of the
parameters in the form \code{c(mu, delta, beta, nu)}.
\item[\code{maxLik}] The value of the maximised log-likelihood.
\item[\code{hessian}] If \code{hessian} was set to \code{TRUE}, the value of
the hessian, not present otherwise.
\item[\code{method}] Optimisation method used.
\item[\code{conv}] Convergence code. See \code{\LinkA{optim}{optim}} or
\code{\LinkA{nlm}{nlm}} for details.
\item[\code{iter}] Number of iterations of optimisation routine.
\item[\code{x}] The data used to fit the distribution.
\item[\code{xName}] Character stirng with the actual \code{x} argument name.
\item[\code{paramStart}] Starting values of the parameters returned by
\code{\LinkA{skewhypFitStart}{skewhypFitStart}}.
\item[\code{svName}] Name of the method used to find starting values.
\item[\code{startValues}] Acronym of method used to find starting values.
\item[\code{breaks}] Cell boundaries found by a call to \code{\LinkA{hist}{hist}}.
\item[\code{midpoints}] The cell midpoints found by a call to
\code{\LinkA{hist}{hist}}.
\item[\code{empDens}] The estimated density found by a call to
\code{\LinkA{hist}{hist}} if \code{startValues = "US"} or
\code{\LinkA{density}{density}} if \code{startValues = "LA"}.
\end{ldescription}
\end{Value}
\begin{Author}\relax
David Scott \email{d.scott@auckland.ac.nz}, Fiona Grimson
\end{Author}
\begin{References}\relax
Aas, K. and Haff, I. H. (2006).
The Generalised Hyperbolic Skew Student's \emph{t}-distribution,
\emph{Journal of Financial Econometrics}, \bold{4}, 275--309.
\end{References}
\begin{SeeAlso}\relax
\code{\LinkA{optim}{optim}}, \code{\LinkA{nlm}{nlm}}, \code{\LinkA{par}{par}},
\code{\LinkA{hist}{hist}}, \code{\LinkA{logHist}{logHist}},
\code{\LinkA{qqskewhyp}{qqskewhyp}}, \code{\LinkA{ppskewhyp}{ppskewhyp}},
\code{\LinkA{dskewhyp}{dskewhyp}} and \code{\LinkA{skewhypFitStart}{skewhypFitStart}}.
\end{SeeAlso}
\begin{Examples}
\begin{ExampleCode}
## See how well skewhypFit works
param <- c(0,1,4,10)
data <- rskewhyp(500, param=param)
fit <- skewhypFit(data)
## Use data set NOK/EUR as per Aas&Haff
data(lrnokeur)
nkfit <- skewhypFit(lrnokeur, method = "nlm")
## Use data set DJI
data(lrdji)
djfit <- skewhypFit(lrdji)
\end{ExampleCode}
\end{Examples}

