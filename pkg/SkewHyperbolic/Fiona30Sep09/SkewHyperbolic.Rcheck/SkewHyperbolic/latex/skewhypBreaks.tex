\HeaderA{skewhypBreaks}{Break points for the Skew Hyperbolic Student's t-Distribuiton}{skewhypBreaks}
\aliasA{ddskewhyp}{skewhypBreaks}{ddskewhyp}
\aliasA{skewhypCalcRange}{skewhypBreaks}{skewhypCalcRange}
\keyword{distribution}{skewhypBreaks}
\begin{Description}\relax
Utility routines that calculate suitable breakpoints for use in
determining the distribution function, and the derivative of the
density function.
\end{Description}
\begin{Usage}
\begin{verbatim}
skewhypBreaks(mu = 0, delta = 1, beta = 1, nu = 1, param =
  c(mu, delta, beta, nu), small = 10^(-6), tiny = 10^(-10),
  deriv = 0.3, ...)
skewhypCalcRange(mu = 0, delta = 1, beta = 1, nu = 1,
  param = c(mu,delta,beta,nu), tol= 10^(-5), ...)
ddskewhyp(x, mu = 0, delta = 1, beta = 1, nu = 1, param =
  c(mu,delta,beta,nu),log = FALSE, tolerance =
  .Machine$double.eps ^ 0.5)
\end{verbatim}
\end{Usage}
\begin{Arguments}
\begin{ldescription}
\item[\code{x}] Vector of quantiles
\item[\code{mu}] Location parameter \eqn{\mu}{}, default is 0.
\item[\code{delta}] Peakedness parameter \eqn{\delta}{}, default is 1.
\item[\code{beta}] Skewness parameter \eqn{\beta}{}, default is 1.
\item[\code{nu}] Shape parameter \eqn{\nu}{}, default is 1.
\item[\code{param}] Specifying the parameters as a vector of the form
\code{c(mu, delta, beta, nu)}.
\item[\code{small}] Size of a small difference between the distribution
function and 0 or 1.
\item[\code{tiny}] Size of a tiny difference between the distribution
function and 0 or 1.
\item[\code{deriv}] Value between 0 and 1 which determines the point at which
the value of the derivative becomes substantial compared to its
maximal value, see \bold{Details}.
\item[\code{log}] Logical; if \code{log = TRUE}, probabilities are given as log(p).
\item[\code{tolerance}] Specified level of tolerance when checking if parameter beta
is equal to 0 in \code{ddskewhyp}.
\item[\code{tol}] Tolerance used in \code{skewhypCalcRange}, see
\bold{Details}.
\item[\code{...}] Passes additional arguments to \code{\LinkA{uniroot}{uniroot}}.
\end{ldescription}
\end{Arguments}
\begin{Value}
\code{ddskewhyp} gives the derivative of \code{\LinkA{dskewhyp}{dskewhyp}}.

\code{skewhypBreaks} returns a list with components:
\begin{ldescription}
\item[\code{xTiny}] Value such that the probabilities to the left are less
than \code{tiny}.
\item[\code{xSmall}] Value such that probabilities to the left are less
than \code{small}.
\item[\code{lowBreak}] Point to the left of the mode such that the
derivative of the density at that point is \code{deriv} times its
maximum value on that side of the mode.
\item[\code{highBreak}] Point to the right of the mode such that the derivative
of the density at that point is \code{deriv} times the value of the
maximum value on that side of the mode.
\item[\code{xLarge}] Value such that the probabities to the right are less than
\code{xSmall}.
\item[\code{xHuge}] Value such that probability to the right is less than
\code{tiny}.
\item[\code{modeDist}] The mode of the given skewed hyperbolic distribution,
calculated by \code{\LinkA{skewhypMode}{skewhypMode}}.
\end{ldescription}


\code{skewhypCalcRange} returns the quantile values at the
endpoints of the region of the density function where the probablity
is greater than \code{tol}.
\end{Value}
\begin{Author}\relax
David Scott \email{d.scott@auckland.ac.nz}, Fiona Grimson
\end{Author}
\begin{References}\relax
Aas, K. and Haff, I. H. (2006).
The Generalised Hyperbolic Skew Student's \emph{t}-distribution,
\emph{Journal of Financial Econometrics}, \bold{4}, 275--309.
\end{References}
\begin{SeeAlso}\relax
\code{\LinkA{uniroot}{uniroot}}, \code{\LinkA{dskewhyp}{dskewhyp}}, \code{\LinkA{skewhypMode}{skewhypMode}}
\end{SeeAlso}
\begin{Examples}
\begin{ExampleCode}
param <- c(0,1,10,10)
range <- skewhypCalcRange(param=param, tol=10^(-3))

#plots of density and derivative
par(mfrow=c(2,1))
curve(dskewhyp(x, param=param), range[1], range[2], n=1000)
title("Density of the Skew\n Hyperbolic Distribution")
curve(ddskewhyp(x, param=param), range[1], range[2], n=1000)
title("Derivative of the Density\n of the Skew Hyperbolic Distribution")

#plot of the density marking the break points
par(mfrow=c(1,1))
range <- skewhypCalcRange(param=param, tol=10^(-6))
curve(dskewhyp(x, param=param), range[1], range[2], n=1000)
title("Density of the Skew Hyperbolic Distribution\n with Breakpoints")
breaks <- skewhypBreaks(param=param)
abline(v=breaks)
\end{ExampleCode}
\end{Examples}

