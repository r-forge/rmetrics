\HeaderA{skewhypMeanVarMode}{Moments and Mode of the Skew Hyperbolic Student's t-Distribution.}{skewhypMeanVarMode}
\aliasA{skewhypKurt}{skewhypMeanVarMode}{skewhypKurt}
\aliasA{skewhypMean}{skewhypMeanVarMode}{skewhypMean}
\aliasA{skewhypMode}{skewhypMeanVarMode}{skewhypMode}
\aliasA{skewhypSkew}{skewhypMeanVarMode}{skewhypSkew}
\aliasA{skewhypVar}{skewhypMeanVarMode}{skewhypVar}
\keyword{distribution}{skewhypMeanVarMode}
\begin{Description}\relax
Functions to calculate the mean, variance, skewness, kurtosis and mode
of a specified skew hyperbolic t-distribution.
\end{Description}
\begin{Usage}
\begin{verbatim}
skewhypMean(mu = 0, delta = 1, beta = 1, nu = 1, param = c(mu, delta,
  beta, nu))
skewhypVar(mu = 0, delta = 1, beta = 1, nu = 1, param = c(mu, delta,
  beta, nu))
skewhypSkew(mu = 0, delta = 1, beta = 1, nu = 1, param = c(mu, delta,
  beta, nu))
skewhypKurt(mu = 0, delta = 1, beta = 1, nu = 1, param = c(mu, delta,
  beta, nu))
skewhypMode(mu = 0, delta = 1, beta = 1, nu = 1, param = c(mu, delta,
  beta, nu), tolerance = .Machine$double.eps ^ 0.5)
\end{verbatim}
\end{Usage}
\begin{Arguments}
\begin{ldescription}
\item[\code{mu}] Location parameter \eqn{\mu}{}, default is 0.
\item[\code{delta}] Peakedness parameter \eqn{\sigma}{}, default is 1.
\item[\code{beta}] Skewness parameter \eqn{\beta}{}, default is 1. Negative
values give a negative skew, positive values give a positive skew.
\item[\code{nu}] Shape parameter \eqn{\nu}{}, default is 1.
\item[\code{param}] Specifying the parameters as a vector of the form
\code{c(mu, delta, beta, nu)}.
\item[\code{tolerance}] A difference smaller than this value is taken to be zero.
\end{ldescription}
\end{Arguments}
\begin{Details}\relax
Users may either specify the values of the parameters individually or
as a vector. If both forms are specified, then the values specified by
the vector \code{param} will overwrite the other ones.

The moments are calculated as per formulae in Aas\&Haff(2006) and the
mode is calculated by numerical optimisation of the density function
using \code{\LinkA{optim}{optim}}.

Note that the mean does not exist when \eqn{\nu = 2}{}, the
variance does not exist for \eqn{\nu \le 4}{}, the skewness does
not exist for \eqn{\nu \le 6}{}, and the kurtosis does not exist
for \eqn{\nu \le 8}{}.
\end{Details}
\begin{Value}
\code{skewhypMean} gives the mean of the skew hyperbolic
t-distribution, \code{skewhypVar} the variance, \code{skewhypSkew} the
skewness, \code{skewhypKurt} the kurtosis and \code{skewhypMode} the
mode.
\end{Value}
\begin{Author}\relax
David Scott \email{d.scott@auckland.ac.nz}, Fiona Grimson
\end{Author}
\begin{References}\relax
Aas, K. and Haff, I. H. (2006).
The Generalised Hyperbolic Skew Student's \emph{t}-distribution,
\emph{Journal of Financial Econometrics}, \bold{4}, 275--309.
\end{References}
\begin{SeeAlso}\relax
\code{\LinkA{dskewhyp}{dskewhyp}}, \code{\LinkA{optim}{optim}}
\end{SeeAlso}
\begin{Examples}
\begin{ExampleCode}
param <- c(10,1,5,9)
skewhypMean(param=param)
skewhypVar(param=param)
skewhypSkew(param=param)
skewhypKurt(param=param)
skewhypMode(param=param)
range <- skewhypCalcRange(param=param)
curve(dskewhyp(x, param=param), range[1], range[2])
abline(v=skewhypMode(param=param), col="red")
abline(v=skewhypMean(param=param), col="blue")
\end{ExampleCode}
\end{Examples}

